\documentclass[10pt,twoside,english,a4paper]{article}
\usepackage[english]{babel}
%\usepackage[T1]{fontenc}
\usepackage[IL2]{fontenc} % lepšia sadzba písmena Ľ než v T1
\usepackage[utf8]{inputenc}
\usepackage{graphicx}
\usepackage{url} % príkaz \url na formátovanie URL
\usepackage{hyperref} % odkazy v texte budú aktívne (pri niektorých triedach dokumentov spôsobuje posun textu)

\usepackage{cite}
%\usepackage{times}

\pagestyle{headings}

\title{Algorithms for improving image processing in mobile photography \thanks{Semestrálny projekt v predmete Metódy inžinierskej práce, ak. rok 2021, vedenie: Artem Vasko}} % meno a priezvisko vyučujúceho na cvičeniach

\author{Artem Vasko\\[2pt]
	{\small Slovenská technická univerzita v Bratislave}\\
	{\small Fakulta informatiky a informačných technológií}\\
	{\small \texttt{xvaskoa1@stuba.sk}}
	}

\date{\small 9. september 2021} % upravte



\begin{document}

\maketitle

% \begin{abstract}

% \end{abstract}



\section{Introduction}

Today slightly every smartphone has a camera. It is very convenient, 
% may be percentage of all photos made on phones vs cameras
until we speak of quality of photos, captured with phone.
By this article I will suggest way to improve quality by some algorithms and
methods for post-processing that can be included to camera
app. I choose this topic because as photograph myself, I was always
unpleasant by my photos taken by phone. It maybe arranges for some people,
but after seeing photos taken on beginner-class camera I struggled on
insufficient quality of taken images. So, this article would add some
possible solutions to the problem of poor dynamic range of photos
taken with mobile devices nowadays.



\section{What are possible solutions?} \label{nejaka}

How can we solve problem with bad quality of photos? There are two ways:
by improving hardware or improving software. But why just don't install
large sensors (with big pixels in it) from the cameras to the smartphones, 
making a huge hardware improvement?
\newline It all because of dimensions of the smartphones. Today it's no
way that big sensors can be handled in smartphone, which width is mostly
limited to  10mm. If you know how camera work, you maybe recently
understand the reason why it so, and it is not the size of sensor, but the
size of the lens. The sensor is not a solar panel-like device whose size
determines how much light you collect. The sensor is behind the lens
and in fact, the lens is the determining factor.
A large diameter lens captures a lot of light.
It almost doesn't matter how large the sensor is. And by installing large
sensor at once we should install lens with huge focal length. This is the
main issue for relatively thin smartphone.
\newline *img of camera structure*
\newline In simple words, larger sensor (with big pixels in it) will require
larger optics in terms of aperture and focal length.
This means that the body of the phone will need to be thicker in
order to provide the required distance between the lens and the sensor.
\newline
% To solve this problem, we need to know the specific reason.
So if we can not just increase the sensors size, we need to go in another
way, namely improve software. The biggest problem is lack of tones and
colors, or in other words, lack  of dynamic range. Especially in
high-contrast photos.
To understand what dynamic range is, we can imagine it as a piano keyboard. There is
a range of tones, in which we can see like piano has keyboard with limited
 width, through which we can play. And the wider
it is, the better. Even human eye can not see all range at once, but sensors 
with big pixels in
digital cameras can because of manual exposure, so we can put exposure
exactly where we want it to be. Only the biggest and most expensive
sensors can accommodate all 12 stops (there is a 12-stop difference in the
highlights and shadows). But we are talking about matrices with small
pixels, by which we want to cover all range of tones.
\newline The simplest decision is to combine few under- and over-exposed
photo into one frame. Doing that we create HDR photo.



% \newline Z obr.~\ref{f:rozhod} je všetko jasné. 

\begin{figure*}[tbh]
	% \includegraphics[scale=1.0]{diagram.pdf}
	\caption{Combining few photos with different exposures 
	usually creates HDR image}
	\label{f:rozhod}
\end{figure*}



\section{Is HDR photos will solve the problem?} \label{ina}
Yes, it would make dynamic range wider. But like every technology 
it has some problems.

Main problem is \ldots{} Firstly lets see some explanation (expl.~\ref{ina:nejake}), 
and then one more (expl.~\ref{ina:este}) to fully understand...
\footnote{Niekedy môžete potrebovať aj poznámku pod čiarou.}

% Môže sa zdať, že problém vlastne nejestvuje\cite{Coplien:MPD}, ale bolo dokázané, že to tak nie je~\cite{Czarnecki:Staged, Czarnecki:Progress}. Napriek tomu, aj dnes na webe narazíme na všelijaké pochybné názory\cite{PLP-Framework}. Dôležité veci možno \emph{zdôrazniť kurzívou}.


\subsection{Explanation} \label{ina:nejake}

Niekedy treba uviesť zoznam:
\begin{itemize}
	\item Dynamic range is the range between the maximum and minimum
	measurable light intensities.
	    %   \begin{itemize}
		      
	    %   \end{itemize}
\end{itemize}


% \begin{enumerate}
% 	\item jedna vec
% 	\item druhá vec
% 	      \begin{enumerate}
% 		      \item x
% 		      \item y
% 	      \end{enumerate}
% \end{enumerate}


\subsection{Explanation} \label{ina:este}

\paragraph{How are made under/over-exposed images}
      \begin{itemize}
		\item Over- and under-exposed images are made by decreasing 
		or respectively increasing shutter-speed in camera.
	      \end{itemize}



\section{Dôležitá časť} \label{dolezita}




\section{Ešte dôležitejšia časť} \label{dolezitejsia}




\section{Záver} \label{zaver} % prípadne iný variant názvu



%\acknowledgement{Ak niekomu chcete poďakovať\ldots}


% týmto sa generuje zoznam literatúry z obsahu súboru literatura.bib podľa toho, na čo sa v článku odkazujete
\bibliography{literatura} \paragraph{https://dl.acm.org/doi/10.5555/2383847.2383888}
\bibliographystyle{plain} % prípadne alpha, abbrv alebo hociktorý iný
\end{document}
